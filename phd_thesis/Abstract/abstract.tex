
% Thesis Abstract -----------------------------------------------------


%\begin{abstractslong}    %uncommenting this line, gives a different abstract heading
\begin{abstracts}        %this creates the heading for the abstract page
The mammalian Target of Rapamycin (mTOR) kinase is a central regulator of cellular growth and metabolism and plays an important role in ageing and age-related diseases. The increase of \emph{in vitro} data collected to extend our knowledge on its regulation, and consequently improve drug intervention, has highlighted the complexity of the mTOR network. This complexity is also aggravated by the intrinsic time-dependent nature of cellular regulatory network cross-talks and feedbacks. Systems biology constitutes a powerful tool for mathematically formalising biological networks and investigating such dynamical properties. \\
The present work discusses the development of three dynamical models of the mTOR network. The first aimed at the analysis of the current literature-based hypotheses of mTOR Complex 2 (mTORC2) regulation. For each hypothesis, the model predicted specific differential dynamics which were systematically tested by \emph{in vitro} experiments. Surprisingly, no current hypothesis could explain the data and a new hypothesis of mTORC2 activation was proposed. The second model extended the previous one with an AMPK module. In this study AMPK was reported to be activated by insulin. Using a hypothesis ranking approach based on model goodness-of-fit, AMPK activity was \emph{in silico} predicted and \emph{in vitro} tested to be activated by the insulin receptor substrate (IRS). Finally, the last model linked mTOR with the oxidative stress response, mitochondrial regulation, DNA damage and FoxO transcription factors. This work provided the characterisation of a dynamical mechanism to explain the state transition from normal to senescent cells and the irreversibility of the senescent phenotype.

%\clearpage
\bigskip
\bigskip
\bigskip
\bigskip

\begin{my_italics_description}{xxxxxxxxxx}
  \normalfont
  \item[\text{Keywords:}] ageing, systems biology, mathematical modelling, mTOR network, insulin signalling, AMPK, cellular senescence, oxidative-stress response, mitochondria, FoxO, Mfn2.
\end{my_italics_description}


\end{abstracts}
%\end{abstractlongs}


% ----------------------------------------------------------------------


%%% Local Variables: 
%%% mode: latex
%%% TeX-master: "../thesis"
%%% End: 
