\graphicspath{{Chapter8/Chapter8Figs/}}

\chapter{Conclusions and outlook}
\label{chap:Conclusions and outlook}
This doctorate thesis presents the dynamical modelling study that I fulfilled for three systems biology-based projects. These works led to the following results:
\begin{enumerate}
 \item\label{Conclusion:project1} mTORC2 can be activated by a PI3K isoform which is sensitive to Wortmannin and independent of p70-S6K-dependent negative feedback loop (NFL);
 \item\label{Conclusion:project2} AMPK can be activated by insulin through insulin receptor substrate (IRS) and dependency on the NFL; and 
 \item\label{Conclusion:project3} formal mechanism underlying the initiation and consolidation of irradiation-induced senescence state with emphasis on the role of the forkhead box family subclass O (FoxO3).
\end{enumerate}
Projects \ref{Conclusion:project1} and \ref{Conclusion:project2} focused on the mammalian target of Rapamycin (mTOR) network (using HeLa and C2C12 cells). Project \ref{Conclusion:project3} provided a comprehensive model of cellular ageing by integrating the mTOR pathway with the DNA-damage and oxidative-stress response, FoxO, and mitochondrial function (using human diploid fibroblast MRC5 cells).\\
Project \ref{Conclusion:project1} advanced our knowledge on the upstream activation of mTORC2 which is of particular importance because: (a) mTORC2 regulates numerous ageing- and cancer-related downstream targets, such as Akt, FoxO, SGK and PKC; (b) a link between mTORC2 and PI3K opens interesting directions on how to pharmacologically reduce PI3K and mTOR complexes activity in a combinatorial manner without unbalancing their downstream signalling function; and (c) the discovery of specific PI3K isoforms responsible for mTORC2 activation, all of which can result in a new major link between mTORC2 and other cellular functions.\\
Project \ref{Conclusion:project2} aimed to extend the previous model with an AMPK module, since mTOR is also regulated by energy, aside from growth factors and nutrients. The finding that AMPK can be regulated by insulin through IRS and that this induction is reduced by the NFL highlights the following main aspects: (a) more study is required in order to understand the significance of AMPK regulation by insulin; (b) new drug interventions may take advantage of this interplay between AMPK and mTORC1, by improving their function and limiting undesired consequences; and (c) important regulatory feedback loops may still be missing and these will notably increase the complexity of the TOR network.\\
Finally, Project \ref{Conclusion:project3} theoretically formalised the mechanism by which normal cells become and stabilise as senescent cells upon irradiation. Importantly, this project shows that: (a) nuclear FoxO plays different roles depending on mode of activation and the characterisation of these must be carefully investigated in order to properly benefit from FoxO activity; (b) combinatorial intervention is necessary to avoid undesired effects in the network; (c) reactive oxygen species (ROS) are an important initiator and contributor for driving ageing, although other components, such as TOR and inflammatory system, also contribute; (d) long term time courses should also be considered for studying the progressive and irreversible state transition during senescence; (e) irradiation-induced senescent state consolidation may be the result of limiting cell damage accumulation in conditions of insufficient energy; (f) the senescent phenotype may be still reversed at later time points if mitochondrial 
function is restored 
(mitochondria re-modelling) and combinatorial intervention to limit internal and external cellular damage accumulation is applied; and (g) since ageing is a state transition, dynamical systems theory should be considered and applied for formalising dynamical models in ageing.\\




%%% ----------------------------------------------------------------------

% ------------------------------------------------------------------------

%%% Local Variables: 
%%% mode: latex
%%% TeX-master: "../thesis"
%%% End: 
